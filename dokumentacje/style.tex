%*******************************************************************************
% Definicje stylu dokumentu
%*******************************************************************************

%===============================================================================
% klasa dokumentu

%\documentclass[12pt, a4paper, twoside, titlepage, final]{mwbk}
%\documentclass[10pt,a4paper,onecolumn,oneside,11pt,wide,floatssmall]{book}
\documentclass[10pt,a4paper]{article}


%===============================================================================
% Pakiety
%\usepackage[latin2]{inputenc}
\usepackage{polski}
%\usepackage[cp1250]{inputenc}
\usepackage[utf8]{inputenc}				% kodowanie �r�d�a
\usepackage[polish]{babel}				% polskie przenoszenie wyraz�w (hyph.)
\usepackage[T1]{fontenc}					% font PL
\usepackage{url}								% polecenie \url
\usepackage{amsfonts}						% fonty matematyczne
\usepackage{graphicx}						% wstawianie grafiki
\usepackage{color}							% kolory
\usepackage{fancyhdr}						% paginy g�rne i dolne
\usepackage[plainpages=false]{hyperref}		% dynamiczne linki
\usepackage{calc}							% operacje arytmetyczne w TeX'u
\usepackage{tabularx}						% rozci�gliwe tabele
\usepackage{array}							% standardowe tabele
\usepackage{geometry}
\usepackage{hyperref}
\usepackage{subfigure}
\usepackage{wrapfig}
\usepackage{indentfirst}
\usepackage{amsmath}
\usepackage{color}
\usepackage{array}
\usepackage{pdflscape}
\usepackage{amsmath}
\usepackage{textcomp}
\usepackage[font={small,it}]{caption}
\usepackage{etoolbox}
\usepackage[section]{placeins}
\usepackage{float}


% \linespread{1.3}								% 1.3 do interlinii 1.5


% \patchcmd{\thebibliography}{\chapter*}{\section*}{}{}

% \bibliographystyle{plain}
% % w�asne pakiety

% %===============================================================================
% % Ustawienia dokumentu

% \frenchspacing

% % ustawienia wymiar�w
% \oddsidemargin 0mm							% margines nieparzystych stron
% \evensidemargin 0mm							% margines parzystych stron
% \headheight 15pt								% wysoko�� paginy g�rnej
% \topmargin 0mm									% margines g�rny
% \setlength{\parindent}{0pt}
% \setlength{\parskip}{1ex plus 0.5ex minus 0.2ex}
% styl paginacji
 \pagestyle{fancy}
% % \renewcommand{\chaptermark}[1]{}%{\markboth{#1}{}} % BO ARTICLE
% \renewcommand{\sectionmark}[1]{}%{\markright{\thesection\ #1}{}}
% \renewcommand{\thesection}{\arabic{section}}


 % naglowek 
 \fancyhf{}
 \fancyhead[R]{\thepage}
 \fancyhead[L]{[BD2]~System wspierający pracę szpitala - faza logiczna}
% %\fancyhead[LO]{\small\nouppercase{\rightmark}}
% %\fancyhead[R]{\small\nouppercase{\leftmark}}
\renewcommand{\headrulewidth}{0.1pt}
\renewcommand{\footrulewidth}{0pt}

% % nag��wek w stylu plain 
\fancypagestyle{plain}
{
\fancyhf{}
\renewcommand{\headrulewidth}{0pt}
\renewcommand{\footrulewidth}{0pt}
}

% % ta sekwencja tworzy czyste kartki na stronach po \cleardoublepage
% \makeatletter
% \def\cleardoublepage{\clearpage\if@twoside \ifodd\c@page\else
% 	\hbox{}
% 	\vspace*{\fill}
% 	\thispagestyle{empty}
% 	\newpage
% 	\if@twocolumn\hbox{}\newpage\fi\fi\fi}
% \makeatother

% %===============================================================================
% % Zmienne �rodowiskowe i polecenia

% % definicja
% % \newtheorem{definition}{Definicja}[chapter] % BO CHAPTER
% \newtheorem{definition}{Definicja}

% % twierdzenie
% % \newtheorem{theorem}{Twierdzenie}[chapter] % BO CHAPTER
% \newtheorem{theorem}{Twierdzenie}

% % obcoj�zyczne nazwy
% \newcommand{\foreign}[1]{\emph{#1}}

% % pozioma linia
% \newcommand{\horline}{\noindent\rule{\textwidth}{0.4mm}}

% % wstawianie obrazk�w {plik}{caption}{opis}
% \newcommand{\fig}[3]
% {
% \begin{figure}[!htb]
% \begin{center}
% \includegraphics[width=\textwidth]{#1}
% \caption[#2]{#2. #3}
% \label{#1}
% \end{center}
% \end{figure}
% }

% %===============================================================================
% % ustawienia pakietu hyperref

% \hypersetup
% {
% %colorlinks=true,			% false: boxed links; true: colored links
% %linkcolor=black,			% color of internal links
% %citecolor=black,			% color of links to bibliography
% %filecolor=black,			% color of file links
% %urlcolor=black			% color of external links
% }

% %===============================================================================